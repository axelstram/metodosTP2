\section{Resultados y discusión}
A continuación, mostramos los resultados obtenidos de la comparación entre los algoritmos PageRank e IN-DEG para instancias particulares. Dichos resultados son la importancia de cada página j, es decir el $x_{j}$. \\

\begin{figure}[H]
\centering     %%% not \center
\subfigure[Figure A]{\label{fig:a}\includegraphics[width=0.4\linewidth]{imagenes/resultadosEstrella.png}}
\subfigure[Figure B]{\label{fig:b}\includegraphics[width=0.3\linewidth]{imagenes/estrella.png}}
\caption{Web con estructura de estrella.}
\end{figure}

En el caso de una web con forma de estrella, cabe destacar las páginas 1 y 2. La página 1 es referenciada por todas las demás, salvo la 2 que es referenciada por la página 1. Dado que IN-DEG definie el ranking de una página j en base a la cantidad de ejes entrantes, es claro que la página uno obtenga la primer posición y la página 2 la segunda posición. Las páginas restantes tienen ranking cero por no ser referenciadas por ninguna otra.
En cambio, para PageRank se hace visible la idea de qué tan importante es el que referencia, en vez de cuantos son los que referencian. Las páginas que no son linkeadas tienen una importancia baja en comparación a las primeras dos.


\begin{figure}[H]
\centering     %%% not \center
\subfigure[Figure A]{\label{fig:a}\includegraphics[width=0.4\linewidth]{imagenes/resultadosCompleto.png}}
\subfigure[Figure B]{\label{fig:b}\includegraphics[width=0.3\linewidth]{imagenes/completo.png}}
\caption{Web con estructura de pentágono.}
\end{figure}

El segundo ejemplo, un grafo completo de cinco nodos, lo hicimos para encontrar los casos en que los criterios de importancia de ambos algoritmos coinciden. Era esperable que todas las páginas tengan la misma importancia en ambos dos, ya que todas las páginas web son referenciadas por páginas con la misma importancia.

\begin{figure}[H]
\centering     %%% not \center
\subfigure[Figure A]{\label{fig:a}\includegraphics[width=0.3\linewidth]{imagenes/resultadosBinario.png}}
\subfigure[Figure B]{\label{fig:b}\includegraphics[width=0.5\linewidth]{imagenes/binario.png}}
\caption{Web con estructura de árbol binario.}
\end{figure}


\begin{figure}[H]
\centering     %%% not \center
\subfigure[Figure A]{\label{fig:a}\includegraphics[width=0.4\linewidth]{imagenes/resultadosCamino.png}}
\subfigure[Figure B]{\label{fig:b}\includegraphics[width=0.2\linewidth]{imagenes/camino.png}}
\caption{Web con estructura de camino.}
\end{figure}

En los dos últimos graficos, se puede apreciar un comportamiento similar a los primeros dos, ya que la estructura subyacente a ambos es similar a las de estos 2 gráficos, generándose rankings similares.	 



\begin{figure}[H]
\centering     %%% not \center
\subfigure[Figure A]{\label{fig:a}\includegraphics[width=0.4\linewidth]{imagenes/resultadosArbolRecursivo.png}}
\subfigure[Figure B]{\label{fig:b}\includegraphics[width=0.4\linewidth]{imagenes/arbolRecursivo.png}}
\caption{Web con estructura de árbol recursivo.}
\end{figure}



\newpage

Posteriormente, ejecutamos los algoritmos de $PageRank$ e $IN-DEG$ para una misma instancia de 500 páginas y 80000 links generada al azar. Queremos ver cuánto cambian los rankings de cada web para cada algoritmo. En el eje X está el puesto de las páginas según $PageRank$, y en el eje Y el valor del ranking de $IN-DEG$ para esa misma página (que está entre 0 y 1) multiplicado por una constante por cuestiones de claridad a la hora de visualizar el gráfico.\\

Podemos ver que en este caso particular, con una matriz bastante esparsa (30\% de los links aproximadamente), ambos algoritmos devuelven un resultado muy similar. Lo que varía es el puesto en PageRank de las webs que tienen pocos links entrantes de diferencia (varía mas en intervalos pequeños). En los casos particulares en donde hay muchas páginas apuntando a una, debería andar peor $IN-DEG$, lo cual no se aprecia en este gráfico ya que es un grafo muy esparso y es aleatorio, pero si en los ejemplos pequeños que vimos anteriormente.\\

\begin{figure}[H]
\centering
\includegraphics[width=0.6\linewidth]{imagenes/pagerankVSindeg.png}
\end{figure}