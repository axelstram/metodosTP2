\section{Desarrollo}

\subsection{PageRank}

El algoritmo PageRank genera un ranking de páginas web de acuerdo a la importancia de cada página, esto es utilizado por los buscadores Web para devolver el resultado más relacionado a la búsqueda realizada por el usuario.\\

Para determinar que página es la más importante, el algoritmo define el puntaje de cada página como la siguiente sumatoria:

\begin{eqnarray}
x_k = \sum_{j \in L_k} \frac{x_j}{n_j},~~~~k = 1,\dots,n.\label{eq:ecuacion1}
\end{eqnarray}

donde $x_{j}$ es la importancia de la página j, $n_{j}$ es la cantidad de links salientes de la página j y $L_{k}$ el conjunto de páginas web con links salientes a j. Dichos puntajes siempre son mayores o iguales a cero, con lo que el puntaje cero corresponde a la página menos importante.\\


Estas ecuaciones pueden ser escritas como un sistema $Ax = x$, con x el vector de los puntajes que buscamos, ya que cada puntaje se obtiene con los de sus páginas vecinas (o sea las del conjunto $L_{k}$).\\

Además, A es una matriz estocástica por columnas, veamos por qué vale esto. Por definición del sistema de ecuaciónes en la ecuación \ref{eq:ecuacion1} vale que

\begin{equation*}
A_{ij} = \left\{
	\begin{array}{cl}
	\frac{1}{n_{j}} & \text{si hay un link de } j \text{ a } i,\\
	0 & \text{en caso contrario.}\\
	\end{array} \right.
\end{equation*}

Con lo cual, la j-ésima columna tiene $n_{j}$ elementos de valor $\frac{1}{n_{j}}$, cuya suma es uno. Con la siguiente proposición, obtenemos que una matriz estocástica por columnas tiene autovalor 1.\\

Proposición: Toda matriz estocástica por columna tiene a uno como autovalor.\\
Demostración: Sea $A \in R^{nxn}$ una matriz estocástica por columna y sea e $\in R^{n}$ un vector columna con todos sus elementos unos. Sabemos que A y $A^{T}$ tienen los mismos autovalores.
Como A es estocástica por columnas, vale que $A^{T} x e = e$ pues las columnas de A (y filas de $A^{T}$) suman uno. Luego como $\lambda = 1$ es autovalor de $A^{T}$ también lo es de A. \\

Luego el problema original equivale a encontrar el autovector x con autovalor 1 para la matriz cuadrada A, es decir, resolver el sistema $Ax = \lambda x$.

Sin embargo, se presentan dos problemas: no sabemos si el autovalor uno tiene multiplicidad uno (o sea, no sabemos si hay una única solución o multiples) y hay problemas con aquellos nodos que no tengan salida (o sea, páginas web que no apuntan a ninguna otra, no tienen salida, con lo que hay columnas de ceros).\\


Para poder conseguir el ranking en el caso de nodos (o sea páginas web) sin links salientes, se suma a aquellas columnas de A con ceros $\frac{1}{n}$ en cada posición de la columna, con lo que se consigue continuar por cualquier página web con la misma probabilidad.
Por otro lado, para evitar los casos en que el autovalor uno tenga multiplicidad mayor a uno, definimos la matriz M de la siguiente manera:

  $$M = (1-m)* A + mS$$

con $m \in (0,1]$ y $S \in R^{nxn}$ una matriz con todos sus elementos iguales a $\frac{1}{n}$.\\
Resolver el sistema $M * x = x$ equivale a 
$$x = M x = (1-m)*A * x+ ms $$

donde s es el vector columna con todas sus entradas $\frac{1}{n}$. Este nuevo sistema asegura que el autovalor uno tiene multiplicidad uno\footnote{Esto puede ser consultado en el paper Bryan, Leise - 2006 - The Linear Algebra behind Google, sección 3.2.}.


Para encontrar el autovector de la matriz correspondiente al autovalor $\lambda$ , aplicamos el algorítmo iterativo de la potencia.\\



\begin{algorithm}
\caption{Método de la Potencia}\label{metpot}
\begin{algorithmic}[1]

  \Function{MetodoPotencia}{Matriz A, vector x, double c, tolerance, maxIter}%\Comment{con $A \in R^{(nxm)*(nxm)}$, $b \in R^{nxm}$}

    \State $x = x/\lVert \mathbf{p} \rVert _{1}$
    \While{ no se alcance la iteracion máxima}
	    \State resolvemos sistema y = A.(1-c)*x + ms
 	    \State obtenemos norma de y
 	    \State 	definimos error como la norma uno de la resta entre x e y
      \If {difieren en menos del error tolerado}
      	\State devolvemos vector y con su norma como autovalor
      \Else
        \State x = y
      \EndIf
    \EndWhile
    \Return false
  \EndFunction

\end{algorithmic}
\end{algorithm}


%Los problemas que se nos presentaron al realizar el algoritmo fueron:
%\begin{itemize}
%\item 
El mayor problema que se nos presentó al realizar el algoritmo fue que en principio dividiamos por la norma infinito, con lo cual el vector resultante no era necesariamente un vector de probabilidades y afectaba el resultado. Releyendo los papers y consultando a docentes lo corregimos por la norma uno.\\
% \end{itemize}


Experimentaciones fallidas y problemas en el transcurso:
\begin{itemize}
\item Al testear el funcionamiento del algoritmo PageRank con la estructura de matriz CSR, notamos que el tiempo de carga era demasiado lento. Esto era por la forma en que funciona esta estructura, al guardar los elementos en orden de filas, si se agregan los elementos en desorden o incluso en orden contrario al orden de filas el tiempo de carga de las matrices de prueba no era razonable.
\end{itemize}


\subsection{GeM}

El algoritmo GeM es un método que está basado en PageRank, con lo cual calcula rankings para cualquier tipo de situación que se pueda representar con un grafo dirigido con pesos en las aristas. Sin embargo, nosotros lo usamos para determinar el ranking para competencias deportivas.\\

GeM representa la competencia deportiva con un grafo dirigido, de manera similar a PageRank. Cada nodo representa un equipo y las aristas representan los enfrentamietos entre los equipos i y j. Con esto se puede representar el grafo con la siguiente matriz:

\begin{equation*}
A_{ji}^t = \left\{
  \begin{array}{cl}
  w_{ji} & \text{si el equipo } i \text{ perdi\'o con el equipo } j,\\
  0 & \text{en caso contrario, }\\
  \end{array} \right.
\end{equation*}

