\section{Conclusiones}
Implementamos dos metodos para resolver muchas instancias de sistemas de ecuaciones en los que variaban solamente los términos independientes. \\
Ordenamos las ecuaciones y los coeficientes de las matrices de manera de poder aprovechar que el valor de temperatura de cada punto de la discretización solo depende de otras cuatro temperaturas, haciendo que las matrices tengan muchos ceros. Por esto las implementaciones de los métodos se pudieron simplificar.\\

Analizamos los tiempos de ejecución de ambos, y cómo varían según el tamaño de los sistemas. Vimos después, de manera gráfica, cómo las temperaturas resultantes son las que esperábamos para el interior del horno. Con algunos ejemplos fue suficiente para ver cómo se dispersa el calor.\\

Para la resolución de un problema como este en que las distintas instancias (a medida que transcurre el tiempo) pueden representarse con sistemas de ecuaciones del tipo AX = b donde la matriz A es siempre la misma, concluímos que el mejor método para usar es la factorización LU. Vimos que la complejidad para los dos métodos es la misma si se quiere resolver un sólo sistema, pero la factorización LU reduce la complejidad cuando se quieren resolver sistemas para muchas instancias del problema. Y más aún, para resolver sistemas más grandes la diferencia se amplía mucho.\\

Como conclusión, nos resultó muy interesante la enorme diferencia temporal entre la eliminación Gaussiana y LU. Creimos en un principio que la cantidad de instancias que se iban a poder correr en un tiempo razonable con una implementacion y en la otra no, iba a ser mucho más pequeña. Realmente se ve una diferencia donde una alternativa puede resultar en que un proyecto se vuelva inviable y otra no (por ejemplo, si quisieramos implementar tal sistema en una planta real con sensores en un horno).\\

Tambien analizamos qué sucede cuando cambia el nivel de discretización, cómo afecta esto a los resultados de los sistemas y a los valores de radios de la isoterma, que era en definitiva lo que buscabamos. Mientras más puntos se toman para calcular las temperaturas, más se acercan los resultados obtenidos a los valores de radios reales de la isoterma. La forma en que se estima el valor de la isoterma cuando no se tiene un resultado preciso es importante, ya que nunca se pueden tomar suficientes puntos de discretización siendo el espacio infinito. Hay que tener en cuenta que mientras más puntos se miden, y entonces más grandes son las matrices, más tiempo se pierde ejecutando los algoritmos.\\

Como pendientes quedaron realizar una optimizacion del uso de la memoria con respecto a la matriz A. También nos quedo pendiente experimentar triangular junto con todos los vectores $b$ como columnas de una matriz, y resolver el sistema de ecuaciones a la vez para todos los $b$, para ver cuánto mejora el tiempo de ejecución de los algoritmos. También quedó pendientene la implementación de la búsqueda binaria para hallar la isoterma. Por el lado de los experimentos, nos hubiera gustado contar con más tiempo para profundizar las conclusiones acerca del caso en que tenemos los valores invertidos en un horno, variando las temperaturas exteriores y las interiores. También pensamos en hacer instancias totalmente aleatorias, pero preferimos usar casos un poco más interesantes y realistas para realizar un análisis sobre los experimentos que elegimos.  
